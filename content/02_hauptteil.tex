\chapterdc{Die Kreiszahl $\pi$}{D}{ie Kreiszahl $\pi$ (auch Ludolphsche Zahl oder Archimedes-Konstante) ist eine reelle mathematische Konstante. Sie beschreibt das Verhältnis des Umfangs eines Kreises zu seinem Durchmesser. Die Bezeichnung $\pi$ leitet sich vom griechischen Wort \emph{perímetros} (Umfang) ab und verweist direkt auf diesen Zusammenhang.\cite{wikipedia:test}}

\subsection{Begriff und Bedeutung}
\subsubsection{Dezimaldarstellung}

Die Zahl $\pi$ besitzt in allen Stellenwertsystemen unendlich viele, nicht periodische Nachkommastellen. Ihre Dezimalentwicklung beginnt mit

\[
\pi = 3{,}14159\,26535\,89793\,23846\,26433\,83279\,50288\,41971\,69399\,37510\,\dots
\]

Für viele praktische Anwendungen wird der Näherungswert $3{,}14$ verwendet.\fncite{stanley:ec02}

\subsection{Mathematische Eigenschaften}

\subsubsection{Irrationalität und Transzendenz}

Die Kreiszahl ist eine irrationale Zahl, sie lässt sich also nicht als Quotient zweier ganzer Zahlen darstellen. Darüber hinaus ist $\pi$ transzendent, das heißt, sie ist keine Lösung einer algebraischen Gleichung mit ganzzahligen Koeffizienten. Diese Eigenschaft hebt sie von vielen anderen irrationalen Zahlen ab, etwa von $\sqrt{2}$, das zwar irrational, aber nicht transzendent ist.\cite{wikipedia:test}

\subsubsection{Bedeutung in verschiedenen Teilgebieten}

$\pi$ tritt in zahlreichen mathematischen Disziplinen auf, unter anderem in

\begin{itemize}
\item der Geometrie (Kreis- und Kugelberechnungen),
\item der Analysis, insbesondere in der Funktionentheorie,
\item der Zahlentheorie und Kombinatorik,
\item der Wahrscheinlichkeitstheorie,
\item sowie in der Physik, etwa bei Schwingungen, Wellen und Kreisbewegungen.
\end{itemize}

Ihre universelle Anwendbarkeit macht $\pi$ zu einer der wichtigsten Konstanten der Mathematik.\cite{wikipedia:test}

\subsection{Historische Entwicklung}

\subsubsection{Antike und Mittelalter}

Bereits in der Antike beschäftigten sich Mathematiker mit der Bestimmung von $\pi$. Um 250 v. Chr. näherte Archimedes die Kreiszahl mithilfe von Vielecken an und konnte sie zwischen $\frac{223}{71}$ und $\frac{22}{7}$ eingrenzen. In China erreichten Liu Hui sowie später Zu Chongzhi durch Polygone mit bis zu 12\,288 Seiten den Näherungsbruch $\frac{355}{113}$, der $\pi$ auf sechs Dezimalstellen genau beschreibt.\cite{wikipedia:test}

\subsubsection{Neuzeitliche Methoden}

Ab dem 16. Jahrhundert wurden in Europa analytische Verfahren entwickelt, die geometrische Methoden zunehmend ablösten. Statt Vielecke zu verwenden, berechnete man $\pi$ mit Hilfe unendlicher Reihen. In der modernen Mathematik kommen für Hochpräzisionsrechnungen Algorithmen wie der Chudnovsky-Algorithmus zum Einsatz.\cite{wikipedia:test}

\subsection{Offene Fragen}

Trotz der langen Forschungsgeschichte sind bis heute nicht alle Eigenschaften der Kreiszahl vollständig geklärt. Eine bedeutende Vermutung ist, dass $\pi$ eine sogenannte \emph{normale Zahl} ist, deren Ziffernfolge statistisch gleichverteilt erscheint. Ein Beweis hierfür steht jedoch noch aus.\cite{wikipedia:test}
